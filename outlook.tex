\section{Outlook}
The results of the conducted measurements show that Raman spectroscopy is in fact a viable option to identificate materials, specifically polymers, and a possible setup has been presented and tested. A qualitative explanation of the phenomenon has been provided.

\bigskip

One could continue to develop the setup in order to get cleaner data, ensuring less background noise and longer integration times, which was not possible due to limits given by the laser sensitivity. Raman spectroscopy could also be used in combination with IR-Spectroscopy to get more complete data.

\bigskip

As mentioned before, Raman spectroscopy is becoming more and more relevant in biochemical, medicinal and environmental sciences as a non-invasive method of identification. In connection to cancer research it is becoming more and more relevant since it can be used to not only identify specific molecules but also differenet tissue types. The identification of organic polymers, such as different plastics, is very relevant and constantly being developed in environmental science to identify mircoplastics.
