\section{Introduction}


The phenomenon of Raman scattering, a change in energy and wavelength of a photon upon interacting with a molecule, was predicted in 1923 by the Austrian physicist Adolf Smekal. It is named after  Indian physicist Chandrasekhara Venkata Raman, often referred to as C.V. Raman, who, along with his students, were the first to observe it and publish papers about it.
\bigskip

In the 37th International young physics tournament (IYPT) Question number 17 is as follows:

\bigskip

7. Quantum Fingerprint:\\
 Shine laser light onto an organic polymer (eg. styrofoam). The scattered light may have a higher or lower wavelength than the incident light. Explain the phenomenon and determine what can be concluded about the molecular structure of the material from the wavelength shift.\cite{iypt}

\bigskip

For the investigation of this problem, a setup was constructed at the physical chemistry laboratory of ETH Zurich to record spectra caused by the emission of photons which were then scattered by an analyte and whose wavelength changed. These changes in wavelength are directly connected to distinct energies that are specific to the analyte.

Then the Raman effects were isolated from the experimental data and then evaluated. A good match between the literature and the experimental data was found. It was possible to assign the peaks in intensity to specific vibrational modes of the given molecules, which made it possible to identify the substances that were analyzed by their Raman spectra.

\bigskip

Raman spectroscopy is used to identify substances in different fields, such as physical chemistry, mineralogy and crystallography, but also pharmaceuticals, clinical medicine and environmental science. It is non-invasive and requires minimal sample preparation, which can be crucial. Its applications are especially growing in clinical medicine in connection to cancer research, and in environmental science in connection to pollution monitoring.

\newpage